\documentclass{beamer}
\usetheme{Copenhagen}
\usepackage{minted}
%\usecolortheme{beetle}
%\usecolortheme{seahorse}
%\usecolortheme{rose}
\usecolortheme[snowy]{owl}
\usetheme{default}

% Title page details
\title{mmuse: Memory management of persistent memory in userspace}
\subtitle{Giving userspace control over dynamic virtual machine guest memory to survive kexec}
\author[James Gowans \& David Woodhouse (EC2) ]{James Gowans (\texttt{jgowans@amazon.com}) \\ \and David Woodhouse (\texttt{dwmw2@infradead.org})}
\institute{Amazon / AWS / EC2}
\date{LPC, MM MC, September 2022}

\begin{document}

\begin{frame}
% Print the title page as the first slide
\titlepage
\end{frame}


\begin{frame}{Agenda}
  \tableofcontents[hideallsubsections]
\end{frame}

% Current section
%\AtBeginSection[ ]
%{
%\begin{frame}{Outline}
%    \tableofcontents[currentsection]
%\end{frame}
%}

\section{Background, Problem and Requirements}

\begin{frame}{Live update}
  Live update of hypervisor via kexec:

  serialise \texttt{->} kexec \texttt{->} deserialise \texttt{->} run\\
    
  Persist guest memory and state across live update (kexec).\\

  Different to snapshot/restore: full restart of userspace process; new VMM binary! Only preserve guest.\\

  Objective: Make live update properly supported! Starting with memory.
\end{frame}

\begin{frame}{Live update memory management}
  Memory ``use cases:''
  \begin{itemize}
    \item Basic case: fixed allocation at launch.
    \item Memory overcommit: dynamic allocation/reclaim incl swap
    \item Deliver faults to userspace for post-copy LM.
    \item Keep DMA running during kexec: IOMMU persistent pgtables
    \item Pass through slice of PCI BAR
    \item Side-car VM: carve out portion of memory to run another VM.
  \end{itemize}
\end{frame}

\section{Implementation Options}
\subsection{Options considered: userspace vs kernel}
\begin{frame}{Options considered}
  \textbf{Fully kernel management persistence:} Traditional kernel-driven allocations.
  Kernel state passed from old to new kernel. 
  Like pkram\footnote{https://lwn.net/Articles/851192/} RFC; think \texttt{tempfs} with persistence.\\
  State can be passed similar to Xen breadcrumbs.\\
  Pros: fast, Cons: complex state hand over.\\
  
  \textbf{Filesystem with userspace control:} Filesystem backed by non-kernel managed memory: hard split of persistent vs ephemeral.
  Provide privileged userspace process the ability to control memory mappings (\texttt{pg\_offset} to \texttt{PFN}) of files.\\
  Store arbitrary file types: memory, state, pgtables, etc.
\end{frame}

\begin{frame}{Suggestion: filesystem with userspace control}
  Idea floated at LSF/MM earlier this year: https://lwn.net/Articles/895453/\\

  Suggesting userspace filesystem. Justification:
  \begin{itemize}
    \item Avoid complexity of passing and re-hydrating state.
    \item Avoid attempting to expose allocation policies and things like swap to persistent memory.
    \item Allow userspace policy and implementation to develop freely.
  \end{itemize}
\end{frame}

\section{Proposal: ``mmuse'' fs (mem mgmt in userspace)}
\begin{frame}{Proposal: ``mmuse'' fs (mem mgmt in userspace)}
  \begin{itemize}
    \item Carve out persistent memory by \texttt{mem=} cmdline param:
    \includegraphics[width=0.5\textwidth]{memmac-memory}
  \item Mount \texttt{mmuse}, setting backing file to something with access to the carved out memory: \texttt{/dev/mem}, \texttt{DAX} device, etc.
    \item Control process: own persistent memory, create files, program allocations into kernel via file ioctls.
    \item Client process (QEMU): use those allocation in non-privileged way: just open the file.
  \end{itemize}
\end{frame}

\begin{frame}[fragile]
  Example: Set up filesystem and backing memory. Mount:

  \begin{minted}{c}
    mount -t mmuse guest-memory /mnt/guest-memory
  \end{minted}
  \vspace{1cm}
  Initially admin file for control process:
  \begin{minted}{c}
    ls -l /mnt/guest-memory/
    -rw-rw---- 1 root root   0 Aug  4 00:00  admin
  \end{minted}
  \vspace{0.8cm}
  Set backing to \texttt{/dev/mem}:
  \begin{minted}{c}
    int admin_fd = open("/mnt/guest-memory/admin")
    int devmem_fd = open("/dev/mem")
    ioctl(admin_fd, SET_BACKING_FD, devmem_fd)
  \end{minted}
\end{frame}

\begin{frame}[fragile]
  Example: Programming a mapping from backing memory to a file:
  \begin{minted}{c}
  dst_fd = open("/mnt/guest-memory/dom123_0_3GiB");
  struct mmuse_mapping mapping = {
    .dst = dst_fd,
    .src_start = 100 * GiB,
    .size = 3 * GiB,
    .dst_start = 0,
    .granulaity = GIGANTIC_PAGE // lvl 3 -> 1 GiB
  };
  ioctl(admin_fd, MAP_MEMORY_RANGE, &dst_fd);
  \end{minted}

  When client processes mmaps that file it would get backing memory, faulting in 1 GiB PTEs.

  Control process would replay after live update.
\end{frame}

\begin{frame}{To persist or not to persist?}
  Should the filesystem preserve state internally across kexec: files, mappings, etc.
  Or should userspace re-drive filesystem state?

  Userspace need to know state anyway, so ought to be able to re-drive.\\

  Advantage of persisting: 1) faster restore on LU, 2) kernel can consume files early.\\

  Advantage of not persisting: No need for memory for pmem for metadata. Generally simpler.\\

  Suggest: no persist at first, retrofit when more stable.
\end{frame}

\begin{frame}{Discussion and Questions}
  Work so far:\\
  Fairly complete proof of concept implemented. Not LKML worthy yet, but could be!\\

  Open floor for questions/comments.\\

  Some ideas for feedback:
  \begin{itemize}
    \item Are we re-inventing or overcomplicating this?
    \item Other use-cases than live-update?
    \item Other ideas to solve this which we should look at?
    \item Should we add \texttt{mmuse} to linux?
  \end{itemize}
\end{frame}

\begin{frame}{Backup slides}
  Backup slides.
\end{frame}

\begin{frame}[fragile]
  Example: heirarchy using one mmuse file as backing memory for another:\\

  \begin{minted}{c}
  int source_fd = open("/mnt/guest_memory/dom:123_memory")

  mount -t mmuse dom:123_memory /mnt/dom:123_memory
  int admin_fd = open("/mnt/dom:123_memory/admin")

  ioctl(admin_fd, SET_BACKING_FD, source_fd)
  \end{minted}
  \vspace{0.7cm}
  Use case: hand over large chunk of memory to guest VMM. That VMM can carve it up for sidecar VMs.
\end{frame}

\begin{frame}[fragile]
  \frametitle{Example: memory overcommit}
  Memory overcommit: reclaim a chunk of memory currently assigned to a file:

  \begin{minted}{c}
  struct mappings = {
    .dst = dst_fd,
    .src_start = (100 << 30),
    .dst_start = 0,
    .size = (16 << 20)
  }
  ioctl(admin_fd, UNMAP_MEMORY_RANGE, &dst_fd);}
  \end{minted}
  \vspace{0.7cm}
  Use case: hand over large chunk of memory to guest VMM. That VMM can carve it up for sidecar VMs.
\end{frame}


\end{document}
